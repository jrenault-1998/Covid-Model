\documentclass[12pt]{article}
\usepackage[utf8x]{inputenc}  
\usepackage{amsmath}
\usepackage{amssymb}
\usepackage{enumerate}
\usepackage[left=2cm,right=2cm,top=2cm,bottom=2cm]{geometry}


\title{COVID-19 Model Containing Contact Tracing and Quarantine}
\date{}

\begin{document} 

\maketitle

\section{Model Equations} \label{equations}

The previous model it is unclear what is meant by quarantining. Quarantining of symptomatic people? Then why are $S$ and $E$ quarantined? Or is instead what is intended self-isolation? If so, why are $I_P$ not potentially placed into self-isolation?

Below is what I suggest for the formulation of contact-tracing:

\begin{eqnarray}
\frac{dS}{dt} &=&  - S a c \left( \frac{I_P + b_CI_C + b_AI_A}{N}\right)  + \delta_{Q_S}  Q_S \\
\frac{dE}{dt} &=& (1-\lambda)S a c \left( \frac{I_P + b_CI_C}{N}\right) + S ac \frac{b_A I_A}{N}  - \delta_E E \\
\frac{dI_P}{dt} &=& r\delta_E E - \delta_{I_P} I_P\\
\frac{dI_C}{dt} &=& \delta_{I_P} I_P - \delta_{I_C} I_C\\
\frac{dI_A}{dt} &=& (1-r) \delta_E E - \delta_{I_A} I_A\\
\frac{dR_S}{dt} &=& \delta_{I_C} I_C +  \delta_{Q_{R_S}} Q_{R_S} \\
\frac{dR_A}{dt} &=& \delta_{I_A} I_A + \delta_{Q_{RA}} Q_{R_A} - \lambda a c R_A \left( \frac{I_P + b_CI_C}{N}\right)  \\
\frac{dQ_S}{dt} &=& \lambda S (1-a) c \left( \frac{I_P + b_CI_C}{N}\right) - \delta_{Q_S}  Q_S \\
\frac{dQ_E}{dt} &=&\lambda S a c \left( \frac{I_P + b_CI_C}{N}\right)  - \delta_E Q_E\\
\frac{dQ_{I_P}}{dt} &=& r \delta_E Q_E -\delta_{I_P} Q_{I_P}\\
\frac{dQ_{I_C}}{dt} &=& \delta_{I_P} Q_{I_P} - \delta_{I_C} Q_{I_C}\\
\frac{dQ_{I_A}}{dt} &=&  (1-r)\delta_E Q_E -\delta_{I_A} Q_{I_A}\\
\frac{dQ_{R_S}}{dt} &=& \delta_{I_C} Q_{I_C} - \delta_{Q_{R_S}} Q_{R_S}\\
\frac{dQ_{R_A}}{ dt} &=& \lambda a c R_A \left( \frac{I_P + b_CI_C}{N}\right) +\delta_{I_A} Q_{I_A} - \delta_{Q_{R_A}}Q_{R_A}
\end{eqnarray}

where $a$ is the probability of infection given a contact, $c$ is the rate of contact between individuals, and $\lambda$ is the probability that an infected individual is placed in quarantine after having been identified as a contact. I recommend drawing out this model to check it and also reading the relevant sections of the Otto \& Day textbook (available at MUN library as an ebook).

One assumption of the above formulation is that the rate of being placed into quarantine after being contacted is the same for $I_P$ and $I_C$, however, this would not be true since $I_P$ is not likely to have yet tested positive. However, correcting this make the system of ODEs very large, and it might be better to move to a system of delay differential equations.

\end{document}